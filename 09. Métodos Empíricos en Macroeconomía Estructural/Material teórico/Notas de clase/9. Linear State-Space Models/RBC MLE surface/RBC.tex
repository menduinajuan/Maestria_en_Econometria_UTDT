%2multibyte Version: 5.50.0.2953 CodePage: 1252

\documentclass[12pt]{article}
%%%%%%%%%%%%%%%%%%%%%%%%%%%%%%%%%%%%%%%%%%%%%%%%%%%%%%%%%%%%%%%%%%%%%%%%%%%%%%%%%%%%%%%%%%%%%%%%%%%%%%%%%%%%%%%%%%%%%%%%%%%%%%%%%%%%%%%%%%%%%%%%%%%%%%%%%%%%%%%%%%%%%%%%%%%%%%%%%%%%%%%%%%%%%%%%%%%%%%%%%%%%%%%%%%%%%%%%%%%%%%%%%%%%%%%%%%%%%%%%%%%%%%%%%%%%
\usepackage{amssymb}
\usepackage{amsmath}
\usepackage{amsfonts}
\usepackage{geophysics}
\usepackage[singlespacing]{setspace}

\setcounter{MaxMatrixCols}{10}
%TCIDATA{OutputFilter=LATEX.DLL}
%TCIDATA{Version=5.50.0.2953}
%TCIDATA{Codepage=1252}
%TCIDATA{<META NAME="SaveForMode" CONTENT="1">}
%TCIDATA{BibliographyScheme=Manual}
%TCIDATA{Created=Thursday, April 07, 2005 15:20:17}
%TCIDATA{LastRevised=Wednesday, August 17, 2016 13:12:43}
%TCIDATA{<META NAME="GraphicsSave" CONTENT="32">}
%TCIDATA{<META NAME="DocumentShell" CONTENT="Articles\SW\Geophysics Journal">}
%TCIDATA{Language=American English}
%TCIDATA{CSTFile=LaTeX article (bright).cst}

\newdimen\dummy
\dummy=\oddsidemargin
\addtolength{\dummy}{72pt}
\marginparwidth=.675\dummy
\marginparsep=.1\dummy
\newtheorem{theorem}{Theorem}
\newtheorem{acknowledgement}[theorem]{Acknowledgement}
\newtheorem{algorithm}[theorem]{Algorithm}
\newtheorem{axiom}[theorem]{Axiom}
\newtheorem{case}[theorem]{Case}
\newtheorem{claim}[theorem]{Claim}
\newtheorem{conclusion}[theorem]{Conclusion}
\newtheorem{condition}[theorem]{Condition}
\newtheorem{conjecture}[theorem]{Conjecture}
\newtheorem{corollary}[theorem]{Corollary}
\newtheorem{criterion}[theorem]{Criterion}
\newtheorem{definition}[theorem]{Definition}
\newtheorem{example}[theorem]{Example}
\newtheorem{exercise}[theorem]{Exercise}
\newtheorem{lemma}[theorem]{Lemma}
\newtheorem{notation}[theorem]{Notation}
\newtheorem{problem}[theorem]{Problem}
\newtheorem{proposition}[theorem]{Proposition}
\newtheorem{remark}[theorem]{Remark}
\newtheorem{solution}[theorem]{Solution}
\newtheorem{summary}[theorem]{Summary}
\newenvironment{proof}[1][Proof]{\noindent\textbf{#1.} }{\ \rule{0.5em}{0.5em}}
\setlength{\oddsidemargin}{0in}
\setlength{\evensidemargin}{0in}
\setlength{\topmargin}{0in}
\setlength{\headheight}{0in}
\setlength{\headsep}{0in}
\setlength{\textheight}{9in}
\setlength{\textwidth}{6.7in}
\input{tcilatex}
\begin{document}


%TCIMACRO{\TeXButton{onehalfspacing}{\onehalfspacing}}%
%BeginExpansion
\onehalfspacing%
%EndExpansion

\section{Basic \textbf{RBC\ Model}}

\begin{equation*}
\max \mathbb{E}_{0}\sum_{t=0}^{\infty }\beta ^{t}\left[ \log c_{t}-\eta 
\frac{l_{t}^{1+\frac{1}{\nu }}}{1+\frac{1}{\nu }}\right] 
\end{equation*}%
subject to%
\begin{equation*}
c_{t}+k_{t+1}=A_{t}k_{t}^{\alpha }l_{t}^{1-\alpha }+\left( 1-\delta \right)
k_{t}
\end{equation*}%
given initial conditions $A_{0},k_{0}$, and a law of motion for the
technology process that we specify below. 

Let $\lambda _{t}$ denote the multiplier on the constraint and write the
Lagrangian%
\begin{equation*}
L=\mathbb{E}_{0}\sum_{t=0}^{\infty }\beta ^{t}\left[ \log c_{t}-\eta \frac{%
l_{t}^{1+\frac{1}{\nu }}}{1+\frac{1}{\nu }}\right] -\lambda _{t}\left[
c_{t}+k_{t+1}-A_{t}k_{t}^{\alpha }l_{t}^{1-\alpha }-\left( 1-\delta \right)
k_{t}\right]
\end{equation*}%
The first order conditions with respect to $c_{t},$ $l_{t},$ and $k_{t+1}$
are, respectively,%
\begin{equation*}
\frac{1}{c_{t}}=\lambda _{t}
\end{equation*}%
\begin{equation*}
\eta l_{t}^{\frac{1}{\nu }}=\lambda _{t}\left( 1-\alpha \right)
A_{t}k_{t}^{\alpha }l_{t}^{-\alpha }
\end{equation*}%
\begin{equation*}
\lambda _{t}=\beta \mathbb{E}_{t}\left[ \lambda _{t+1}\left( \alpha
A_{t+1}k_{t+1}^{\alpha -1}l_{t+1}^{1-\alpha }+1-\delta \right) \right] .
\end{equation*}%
plus the feasibility constraint. The transversality condition of this
problem is%
\begin{equation*}
\lim_{T\rightarrow \infty }E_{0}\left[ \beta ^{T}\lambda _{T}k_{T+1}\right]
=0.
\end{equation*}

\underline{\textbf{Shocks}}\textbf{. }The logarithm of TFP follows an AR(1)
process%
\begin{equation*}
\log A_{t+1}=\rho \log A_{t}+\varepsilon _{t+1}
\end{equation*}%
where $\varepsilon _{t+1}$ is i.i.d. normal with mean 0 and variance $\sigma
_{\varepsilon }^{2}$.

\bigskip

\underline{\textbf{Equilibrium equations}}

We can write the equilibrium conditions as the following system of 7
equations%
\begin{equation}
\frac{1}{c_{t}}=\lambda _{t}  \label{eq1}
\end{equation}%
\begin{equation}
\eta l_{t}^{\frac{1}{\nu }}=\lambda _{t}\left( 1-\alpha \right) \frac{y_{t}}{%
l_{t}}  \label{eq2}
\end{equation}%
\begin{equation}
\lambda _{t}=\beta \mathbb{E}_{t}\left[ \lambda _{t+1}\left( \alpha \frac{%
y_{t+1}}{k_{t+1}}+1-\delta \right) \right]   \label{eq3}
\end{equation}%
\begin{equation}
y_{t}=A_{t}k_{t}^{\alpha }l_{t}^{1-\alpha }  \label{eq4}
\end{equation}%
\begin{equation}
c_{t}+x_{t}=y_{t}  \label{eq5}
\end{equation}%
\begin{equation}
x_{t}=k_{t+1}-\left( 1-\delta \right) k_{t}  \label{eq6}
\end{equation}%
\begin{equation}
\log A_{t+1}=\rho \log A_{t}+\varepsilon _{t+1}  \label{eq7}
\end{equation}

\subsection{Steady state}

The second step in the procedure consists of finding the non-stochastic
steady steate of the economy and calibrating the model. In steady state the
system (\ref{eq1})-(\ref{eq7}) becomes%
\begin{equation}
\frac{1}{\bar{c}}=\bar{\lambda}  \label{ss1}
\end{equation}%
\begin{equation}
\eta \bar{l}^{\frac{1}{\nu }}=\bar{\lambda}\left( 1-\alpha \right) \bar{y}/%
\bar{l}  \label{ss2}
\end{equation}%
\begin{equation}
1=\beta \left( \alpha \bar{y}/\bar{k}+1-\delta \right)  \label{ss3}
\end{equation}%
\begin{equation}
\bar{y}=\bar{A}\bar{k}^{\alpha }\bar{l}^{1-\alpha }  \label{ss4}
\end{equation}%
\begin{equation}
\bar{c}+\bar{x}=\bar{y}  \label{ss5}
\end{equation}%
\begin{equation}
\bar{x}=\delta \bar{k}  \label{ss6}
\end{equation}%
\begin{equation}
\bar{A}=1.  \label{ss7}
\end{equation}

The system can be reduced to (get rid of $\bar{\lambda}$, $\bar{x}$)%
\begin{equation*}
\eta \bar{l}^{\frac{1}{\nu }}=\frac{1}{\bar{c}}\left( 1-\alpha \right) \frac{%
\bar{y}}{\bar{l}}
\end{equation*}%
\begin{equation*}
1=\beta \left( \alpha \frac{\bar{y}}{\bar{k}}+1-\delta \right) 
\end{equation*}%
\begin{equation*}
\frac{\bar{y}}{\bar{k}}=\left( \frac{\bar{l}}{\bar{k}}\right) ^{1-\alpha }
\end{equation*}%
\begin{equation*}
\frac{\bar{c}}{\bar{k}}+\delta =\frac{\bar{y}}{\bar{k}}
\end{equation*}%
From the second equation we find $\bar{y}/\bar{k}:$%
\begin{equation*}
\frac{\bar{y}}{\bar{k}}=\frac{1/\beta -\left( 1-\delta \right) }{\alpha }
\end{equation*}%
and the third then gives $\bar{l}/\bar{k}:$%
\begin{equation*}
\frac{\bar{l}}{\bar{k}}=\left( \frac{\bar{y}}{\bar{k}}\right) ^{\frac{1}{%
1-\alpha }}\Rightarrow \frac{\bar{l}}{\bar{k}}=\left( \frac{1/\beta -\left(
1-\delta \right) }{\alpha }\right) ^{\frac{1}{1-\alpha }}
\end{equation*}%
The last equation then gives $\bar{c}/\bar{k}$%
\begin{equation*}
\frac{\bar{c}}{\bar{k}}=\frac{\bar{y}}{\bar{k}}-\delta \Rightarrow \frac{%
\bar{c}}{\bar{k}}=\frac{1/\beta -\left( 1-\delta \right) }{\alpha }-\delta 
\end{equation*}%
So we are left with the last equation.\ We write it as%
\begin{equation*}
\eta \bar{l}^{\frac{1}{\nu }+1}=\frac{\bar{k}}{\bar{c}}\left( 1-\alpha
\right) \frac{\bar{y}}{\bar{k}}
\end{equation*}%
Since we know $\bar{c}/\bar{k}$ and $\bar{y}/\bar{k}$, we know the right
side. From here we solve for $\bar{l}:$%
\begin{equation}
\bar{l}=\left( \frac{1-\alpha }{\eta }\frac{\bar{y}/\bar{k}}{\bar{c}/\bar{k}}%
\right) ^{\frac{1}{1+1/\nu }}  \label{lbar}
\end{equation}%
Once we have $\bar{l},$ we recover $\bar{k}$ from $\bar{l}/\bar{k}$, and
then we have the entire steady state

\subsection{Log-linearization of the model}

Log-linearized model%
\begin{eqnarray}
0 &=&\hat{c}_{t}+\hat{\lambda}_{t}  \label{lin eq1} \\
0 &=&(1+\frac{1}{\nu })\hat{l}_{t}-\hat{\lambda}_{t}-\hat{y}_{t}
\label{lin eq2} \\
0 &=&\hat{y}_{t}-\hat{A}_{t}-\alpha \hat{k}_{t}-\left( 1-\alpha \right) \hat{%
l}_{t}  \label{lin eq3} \\
0 &=&\bar{y}\hat{y}_{t}-\bar{c}\hat{c}_{t}-\bar{x}\hat{x}_{t}
\label{lin eq4} \\
\mathbb{E}_{t}[\hat{k}_{t+1}] &=&\left( 1-\delta \right) \hat{k}_{t}+\delta 
\hat{x}_{t}  \label{lin eq5} \\
\mathbb{E}_{t}\left[ \hat{\lambda}_{t+1}+\beta \alpha \left( \bar{y}/\bar{k}%
\right) \left( \hat{y}_{t+1}-\hat{k}_{t+1}\right) \right]  &=&\hat{\lambda}%
_{t}  \label{lin eq6} \\
\mathbb{E}_{t}\left[ \hat{A}_{t+1}\right]  &=&\rho \hat{A}_{t}.
\label{lin eq7}
\end{eqnarray}%
Note that I wrote $\mathbb{E}_{t}[\hat{k}_{t+1}]$ even though $\hat{k}_{t+1}$
is chosen (and therefore already known) at time $t$. This is just notation
that will allows us to write the model as the following first order vector
expectational difference equation%
\begin{equation}
\mathbf{A}\mathbb{E}_{t}\left[ \mathbf{z}_{t+1}\right] =\mathbf{Bz}_{t}
\label{Linear Diff Eqn}
\end{equation}%
where the vector $\mathbf{z}_{t}$ contains all the variables in the economy
and $\mathbf{A}$ and $\mathbf{B}$ are square matrices.

We solve numerically this model using the Matlab program \texttt{solab.m}.
We order the variables $\mathbf{z}_{t}$ as follows:%
\begin{equation*}
\mathbf{z}_{t}=\left[ 
\begin{array}{c}
\text{endogenous states variables} \\ 
\text{exogenous states variables} \\ 
\text{jump variables}%
\end{array}%
\right]
\end{equation*}%
In the RBC\ model described above, the only endogenous state variable is the
stock of capital $\hat{k}_{t}$ and the only exogenous state variable is the
level of technology $\hat{A}_{t}$.\ Therefore, the variable $\mathbf{z}_{t}$
is given by%
\begin{equation}
\mathbf{z}_{t}=\left[ \hat{k}_{t},\ \hat{A}_{t},\ \ \hat{y}_{t},\ \hat{c}%
_{t},\ \hat{l}_{t},\ \hat{x}_{t},\ \hat{\lambda}_{t}\right] ^{\prime }.
\label{ordering z}
\end{equation}%
Please note that the order within each group of variables does not matter
(e.g. we could put $c_{t}$ before $y_{t}$ in the vector $\mathbf{z}_{t}$).

In addition, we must tell the program how many of the variables in $\mathbf{z%
}_{t}$ are state variables. In our case, it is 2: $\hat{k}_{t}$ and $\hat{A}%
_{t}$. Note that, in this case, $\mathbf{A}$ and $\mathbf{B}$ are $7\times 7$
matrices. If we let $\mathbf{\kappa }_{t}\equiv \lbrack \hat{k}_{t},\hat{A}%
_{t}]^{\prime }$ denote the vector of state variables and $\mathbf{u}_{t}=[%
\hat{y}_{t},\ \hat{c}_{t},\ \hat{l}_{t},\ \hat{x}_{t},\ \hat{\lambda}_{t}]$,
the vector of jump variables, the solver delivers the equilibrium of the
\textquotedblleft certainty equivalent\textquotedblright\ model in the form%
\begin{eqnarray*}
\mathbf{u}_{t} &=&\mathbf{F\kappa }_{t} \\
\mathbf{\kappa }_{t+1} &=&\mathbf{P\kappa }_{t}
\end{eqnarray*}%
The \textquotedblleft stochastic\textquotedblright\ solution of the model is
obtained by replacing the second equation above with%
\begin{equation*}
\mathbf{\kappa }_{t+1}=\mathbf{P\kappa }_{t}+\left[ 
\begin{array}{c}
0 \\ 
1%
\end{array}%
\right] \varepsilon _{t+1}
\end{equation*}%
which simply recovers the stochastic shock $\hat{A}_{t+1}=\rho \hat{A}%
_{t}+\varepsilon _{t+1}$.

For the ordering (\ref{ordering z}), the matrices $\mathbf{A}\mathbb{\ }$and 
$\mathbf{B}$ of the system (\ref{Linear Diff Eqn}) are given by%
\begin{equation*}
\mathbf{A=}\left[ 
\begin{array}{ccccccc}
0 & 0 & 0 & 0 & 0 & 0 & 0 \\ 
0 & 0 & 0 & 0 & 0 & 0 & 0 \\ 
0 & 0 & 0 & 0 & 0 & 0 & 0 \\ 
0 & 0 & 0 & 0 & 0 & 0 & 0 \\ 
1 & 0 & 0 & 0 & 0 & 0 & 0 \\ 
-\beta \alpha \left( \bar{y}/\bar{k}\right) & 0 & \beta \alpha \left( \bar{y}%
/\bar{k}\right) & 0 & 0 & 0 & 1 \\ 
0 & 1 & 0 & 0 & 0 & 0 & 0%
\end{array}%
\right] \rightarrow \left[ 
\begin{array}{c}
\text{equation (\ref{lin eq1})} \\ 
\text{equation (\ref{lin eq2})} \\ 
\text{equation (\ref{lin eq3})} \\ 
\text{equation (\ref{lin eq4})} \\ 
\text{equation (\ref{lin eq5})} \\ 
\text{equation (\ref{lin eq6})} \\ 
\text{equation (\ref{lin eq7})}%
\end{array}%
\right]
\end{equation*}%
\begin{equation*}
\mathbf{B=}\left[ 
\begin{array}{ccccccc}
0 & 0 & 0 & 1 & 0 & 0 & 1 \\ 
0 & 0 & -1 & 0 & (1+\frac{1}{\nu }) & 0 & -1 \\ 
-\alpha & -1 & 1 & 0 & -\left( 1-\alpha \right) & 0 & 0 \\ 
0 & 0 & \bar{y} & -\bar{c} & 0 & -\bar{x} & 0 \\ 
1-\delta & 0 & 0 & 0 & 0 & \delta & 0 \\ 
0 & 0 & 0 & 0 & 0 & 0 & 1 \\ 
0 & \rho & 0 & 0 & 0 & 0 & 0%
\end{array}%
\right] \rightarrow \left[ 
\begin{array}{c}
\text{equation (\ref{lin eq1})} \\ 
\text{equation (\ref{lin eq2})} \\ 
\text{equation (\ref{lin eq3})} \\ 
\text{equation (\ref{lin eq4})} \\ 
\text{equation (\ref{lin eq5})} \\ 
\text{equation (\ref{lin eq6})} \\ 
\text{equation (\ref{lin eq7})}%
\end{array}%
\right]
\end{equation*}

Using the calibrated parameter values, the model delivers the following
solution:%
\begin{equation*}
\mathbf{F=}\left[ 
\begin{array}{cc}
0.22 & 1.33 \\ 
0.57 & 0.34 \\ 
-0.17 & 0.50 \\ 
-1.10 & 5.07 \\ 
-0.57 & -0.34%
\end{array}%
\right] ;\ \ \ \mathbf{P=}\left[ 
\begin{array}{cc}
0.96 & 0.09 \\ 
0 & 0.95%
\end{array}%
\right]
\end{equation*}%
Which, in other words, implies the following policy functions:%
\begin{eqnarray*}
\hat{y}_{t} &=&0.22\hat{k}_{t}+1.33\hat{A}_{t} \\
\hat{c}_{t} &=&0.57\hat{k}_{t}+0.34\hat{A}_{t} \\
\hat{l}_{t} &=&-0.17\hat{k}_{t}+0.50\hat{A}_{t} \\
\hat{x}_{t} &=&-1.10\hat{k}_{t}+5.07\hat{A}_{t} \\
\hat{k}_{t+1} &=&0.96\hat{k}_{t}+0.09\hat{A}_{t} \\
\hat{A}_{t+1} &=&0.95\hat{A}_{t}+\varepsilon _{t+1}.
\end{eqnarray*}

Once we have this solution, we can compute impulse responses, variance
decompositions, simulations, compute spectral densities, and so forth.

\end{document}
